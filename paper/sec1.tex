\subsection*{Subsection}

A significant body of research exploring the relationship between the environment and physical activity has developed in diverse fields such as urban planning, economics, criminology, transportation, psychology, exercise science, and public health.  Research ranges from macro-scale studies of transportation mode choices across cities and regions to small scale investigations of specific types of walking behavior among small subpopulations.  In public health, efforts have been made to determine correlates of purposeful physical activity such as walking or bicycling to school or work and the environment, since active transport is seen as a potential key to increasing physical activity and lowering rates of obesity and related chronic diseases.  
This is especially true for youth, for whom rates of physical activity have continued to fall as obesity rises. Additional research has focused on the relationship between active transport and overall levels of physical activity.
  
\subsection*{Another subsection} %Brief lit review of field

Bla bla bla this is an example of a citation \cite{joe_dependence_2015}

We can also include a picture as seen in figure \ref{Fi:some_fig}. it's a placeholder for now

\begin{figure}
\begin{center}
\includegraphics[scale=.7]{fig/placeholder}
\caption{Some Figure}
\label{Fi:some_fig}
\end{center}
\end{figure}

\subsection*{Last Subsection}

Multiple Opportunities to Reach Excellence (MORE) is a longitudinal, epidemiological study designed to assess the relationship between youth and long term exposure violence. The study participants are children in 4th and 5th grades in six Baltimore City public schools. In-person interviews were conducted with participating children and their parents to collect data on demographics, child activity (including child and parent reported walking to school behavior), and perceptions of the surrounding neighborhood.

\begin{gather}
q_{j}=\frac{\sum_{i}b_{i}p_{i}}{\sum_{i}l_{i}p_{i}}
\end{gather}

Data was collected from 365 children in August through October 2007 as part of cohort 1 of the MORE study, however only 362 of the study participants had addresses which were successfully geocoded \cite{nelsen_introduction_2006}.
