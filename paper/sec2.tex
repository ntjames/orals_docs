\subsection*{Copula families}

Much more background on copulas

Elliptical

Archimedean

Combinations/Comprehensive

Here is some additional text and we can also reference a table (see table \ref{Ta:a_table}).  The incivility was the sum of features present within a block \cite{costa_case_2017}.

\begin{table}
\begin{center}
\caption{Popular Copulas}
\begin{tabular}{cc}
  \hline \\
Independence & Normal \\
t-Copula &other elliptical?? \\
Archimedean  &Clayton\\
\multicolumn{2}{c}{Comprehensive}\\ \\
  \hline
\end{tabular}
\label{Ta:a_table}
\end{center}
\end{table}


\subsection*{Copula regression}

This is a whole section on copula regression
\begin{gather}
E(Z(s))=\mu(s)=\mu,\text{for all }s \in D \label{E:2.1}\\
Var(Z(s))=\sigma^{2}(s)=\sigma^{2},\text{for all }s \in D \label{E:2.2}\\
Cov(Z(s_{1}),Z(s_{2}))=C(s_{i}-s_{j}),\text{for all }s_{i}\ne s_{j} \in D \label{E:2.3}
\end{gather}
Equations \ref{E:2.1} and \ref{E:2.2} ensure that the mean and variance are constant and independent of location throughout the region $D$. Equation \ref{E:2.3} ensures that the covariance depends only on the difference between two locations, rather than the locations themselves. $C(\cdot)$ in equation \ref{E:2.3} is called the covariance function.

 (\cite{smith_bayesian_nodate}, \cite{biswas_bayesian_2009}).  
 
 We use the R package ``copula" \cite{hofert_copula:_2016}. 
