\documentclass{beamer}
\usepackage{amsfonts, amssymb, amsmath, amsthm, graphicx, multirow, beamerthemesplit}

\title[Copula Modeling for Clinical Trials]{Copula Modeling for Clinical Trials}
\author{Nathan T. James, ScM}
\institute{Dept. of Biostatistics, Vanderbilt University}
\date{December 14, 2018}

\begin{document}
	
	\frame{\titlepage}	
%	\frame{\tableofcontents}
	
\section{Introduction}

	\begin{frame}
		\frametitle{Introduction}
		\begin{itemize}
			\item Ordinal Cumulative Probability Model (CPM)
	           \begin{itemize}
		           	\item $G[P(Y \le y_i|X)]=\alpha_i-\beta^{T}X$ 
	           		\item $y_i$ - ordered, continuous outcome
	           		\item $X$ - matrix of covariates
	           		\item $G(\cdot)$ - link function 
	           \end{itemize}
	     \end{itemize}
	     Why use a Bayesian CPM with a \textbf{continuous} outcome?
	     \begin{itemize}
	        \item Invariant to monotonic transformation of outcome 
	        \item Directly model full conditional CDF 
           	\item Handles any ordered outcome including mixed discrete/continuous distributions (e.g., continuous outcome with lower limit of detection) 
           	\item Inference using posterior probabilities   	
		\end{itemize}
	\end{frame}
	
\section{Model Interpretation}
	
	\begin{frame}
		\begin{itemize}
		\item $\alpha_i$  estimate posterior CDF for $X=0$
		\item $\beta$ measure association between $X$ and distribution of $Y$; interpretation depends on link function 
        \item Mean and quantiles calculated from posterior distribution of full conditional CDF using single model
		\end{itemize}
		\begin{center}
			\includegraphics[scale=0.53]{fig/placeholder}
		\end{center}
	\end{frame}
	
\section{Model Performance}

	\begin{frame}
		\begin{itemize}
	%	\item software comparison
		\item Implemented using \texttt{brms} and \texttt{rstanarm}; both call \texttt{Rstan}
		\item Different parameterizations; using default priors \texttt{rstanarm} more accurate in simulations 
		\item Model convergence depends on package and link function
	%	\item Ex. logistic model, logit data, n=100
		\end{itemize}
		\begin{center}
		%	\includegraphics[scale=.55]{../fig/mod_comp1}		
		\end{center}
	\end{frame}

	\begin{frame}
		% Computational Efficiency
		\begin{itemize}
			\item \texttt{brms} needs to compile C++ code, \texttt{rstanarm} pre-compiled
			\item Major differences in computation time based on link function
			\item For datasets up to $\sim 1000$ distinct $y$ values
			computation time is approximately linear for both packages; for larger datasets compute time increases at a faster rate for \texttt{brms}
		\end{itemize}
		\begin{center}
		%	\includegraphics[scale=.51]{../fig/comptime_accre}
		\end{center}
		\end{frame}

		\begin{frame}
		\begin{itemize}
		%	\item Model Misspecification
			\item With moderate sample size, reasonably robust to misspecification of link function
			\item Uncertainty in link function can be accounted for using a mixture of links
		\end{itemize}			
		\begin{center}
		%	\includegraphics[scale=.43]{../fig/misspec_plot1}
		\end{center}
		\end{frame}
	
	\begin{frame}
	Contact\\
	email: nathan.t.james AT vanderbilt.edu\\
	web: ntjames.com\\
	twitter: @nalhsyjones
	\end{frame}

\end{document} 